% REMEMBER: You must not plagiarise anything in your report. Be extremely careful.

\documentclass{l4proj}

    
%
% put any additional packages here
%
\renewcommand\floatpagefraction{.9}
\renewcommand\topfraction{.9}
\renewcommand\bottomfraction{.9}
\renewcommand\textfraction{.1}   
\setcounter{totalnumber}{50}
\setcounter{topnumber}{50}
\setcounter{bottomnumber}{50}

\begin{document}

%==============================================================================
%% METADATA
\title{Context Relevant Anti-Phishing Training }
\author{Joel Feven}
\date{September 14, 2022}

\maketitle

%==============================================================================
%% ABSTRACT
\begin{abstract}
    Every abstract follows a similar pattern. Motivate; set aims; describe work; explain results.
    \vskip 0.5em
    ``XYZ is bad. This project investigated ABC to determine if it was better. 
    ABC used XXX and YYY to implement ZZZ. This is particularly interesting as XXX and YYY have
    never been used together. It was found that  
    ABC was 20\% better than XYZ, though it caused rabies in half of subjects.''
\end{abstract}

%==============================================================================

% EDUCATION REUSE CONSENT FORM
% If you consent to your project being shown to future students for educational purposes
% then insert your name and the date below to  sign the education use form that appears in the front of the document. 
% You must explicitly give consent if you wish to do so.
% If you sign, your project may be included in the Hall of Fame if it scores particularly highly.
%
% Please note that you are under no obligation to sign 
% this declaration, but doing so would help future students.
%
%\def\consentname {My Name} % your full name
%\def\consentdate {20 March 2018} % the date you agree
%
\educationalconsent


%==============================================================================
\tableofcontents

%==============================================================================
%% Notes on formatting
%==============================================================================
% The first page, abstract and table of contents are numbered using Roman numerals and are not
% included in the page count. 
%
% From now on pages are numbered
% using Arabic numerals. Therefore, immediately after the first call to \chapter we need the call
% \pagenumbering{arabic} and this should be called once only in the document. 
%
% The first Chapter should then be on page 1. You are allowed 40 pages for a 40 credit project and 20 pages for a 
% 20 credit report. This includes everything numbered in Arabic numerals (excluding front matter) up
% to but excluding the appendices and bibliography.
%
% You must not alter text size (it is currently 10pt) or alter margins or spacing.
%
%
%==================================================================================================================================
%
% IMPORTANT
% The chapter headings here are **suggestions**. You don't have to follow this model if
% it doesn't fit your project. Every project should have an introduction and conclusion,
% however. 
%
%==================================================================================================================================
\chapter{Introduction}

% reset page numbering. Don't remove this!
\pagenumbering{arabic} 


Why should the reader care about what are you doing and what are you actually doing?
\section{Guidance}

\textbf{Motivate} first, then state the general problem clearly. 

\section{Writing guidance}
\subsection{Who is the reader?}

This is the key question for any writing. Your reader:

\begin{itemize}
    \item
    is a trained computer scientist: \emph{don't explain basics}.
    \item
    has limited time: \emph{keep on topic}.
    \item
    has no idea why anyone would want to do this: \emph{motivate clearly}
    \item
    might not know \emph{anything} about your project in particular:
    \emph{explain your project}.
    \item
    but might know precise details and check them: \emph{be precise and
    strive for accuracy.}
    \item
    doesn't know or care about you: \emph{personal discussions are
    irrelevant}.
\end{itemize}

Remember, you will be marked by your supervisor and one or more members
of staff. You might also have your project read by a prize-awarding
committee or possibly a future employer. Bear that in mind.

\subsection{References and style guides}
There are many style guides on good English writing. You don't need to
read these, but they will improve how you write.

\begin{itemize}
    \item
    \emph{How to write a great research paper} \cite{Pey17} (\textbf{recommended}, even though you aren't writing a research paper)
    \item
    \emph{How to Write with Style} \cite{Von80}. Short and easy to read. Available online.
    \item
    \emph{Style: The Basics of Clarity and Grace} \cite{Wil09} A very popular modern English style guide.
    \item
    \emph{Politics and the English Language} \cite{Orw68}  A famous essay on effective, clear writing in English.
    \item
    \emph{The Elements of Style} \cite{StrWhi07} Outdated, and American, but a classic.
    \item
    \emph{The Sense of Style} \cite{Pin15} Excellent, though quite in-depth.
\end{itemize}

\subsubsection{Citation styles}

\begin{itemize}
\item If you are referring to a reference as a noun, then cite it as: ``\citet{Orw68} discusses the role of language in political thought.''
\item If you are referring implicitly to references, use: ``There are many good books on writing \citep{Orw68, Wil09, Pin15}.''
\end{itemize}
\subsection{Plagiarism warning}

\begin{highlight_title}{WARNING}
    
    If you include material from other sources without full and correct attribution, you are commiting plagiarism. The penalties for plagiarism are severe.
    Quote any included text and cite it correctly. Cite all images, figures, etc. clearly in the caption of the figure.
\end{highlight_title}


%==================================================================================================================================
\chapter{Background}

\section{What is Phishing?}
With the rise of technology and computer-based solutions, many corporations around the world rely on computers for data storage, analysis, and communication. However, this has one fatal flaw: people are required to operate the vast majority of this technology. This has led to a large number of cyber-attacks directed towards the person behind the screen rather than the device itself, mainly through an attack known as “phishing”.

Phishing is a type of attack that makes use of social engineering to coerce users into giving personal information over to these attackers, known as “phishers”. This can be done with varying degrees of complexity, from generic widespread “spam” emails to user specific, personalised attacks (also known as “spear phishing”). When spear phishing targets a highly influential person within a company, this can be known as “whaling” due to the large return of data/credentials/access from the company. In Q4 2016, the Anti-Phishing Working Group (APWG) found that each month there were around 90,000 phishing attacks which was an increase of 5753\% of average phishing attacks from the last 12 years \citep{apwg2017phishing}. However, phishing is not limited to exclusively emails. Phishing can be performed in a wide variety of methods and on different platforms such as social media and instant messaging services. As long as the user is exploited to perform an action or relay sensitive information, the attack is considered a phishing attack.

\subsection{Tactics}
\label{sec:other_tactics}

Phishing often makes use of vulnerabilities within browsers. By convincing the user to perform a specified action, this can enable the phisher to take advantage of a number of exploits. URL links are commonly used in phishing emails, often with an urgent prompt for the user to click the link. These URLs can be obfuscated by phishers through several means, although most often it is through sound-squatting and typo-squatting \citep{chiew2018survey}. Both are methods of URL manipulation to trick the user into thinking the URL is recognisable and/or legitimate. Sound-squatting is the technique of changing the domain names of websites to sound similar to legitimate domains, often making use of homonyms to confuse the user (e.g. www.ryanairways.com to www.ryanheirways.com). Typo-squatting uses a similar technique but uses domains with intentional typos to trick the user (e.g. www.anazon.com). This can also be used to “phish” users that genuinely make a typo when entering the URL address of a legitimate website.

These links will redirect users to compromised webpages that can take advantage of the user’s device in a number of ways. Websites can use cross-site scripting attacks and SQL injections to inject harmful code into the user’s device to potentially access data stored. However, the entire website doesn’t not need to be custom made by the phisher and instead the malicious URL could be used to impose a malicious element on top of the legitimate website, this is known as HTML substitution \citep{chiew2018survey}.

\subsection{Targets of Phishing}

\subsection{Why is Phishing Worth Investigating?}

\section{Context in Phishing}

The most successful types of phishing are often the most personalised, context-based attacks. These are often aimed at very specific people/departments within major companies making use of context-relevant names and details. For example, one such attack is the 2010 Kneber botnet attacks on government systems \citep{villeneuve2010kneber}. These attacks made use of security-based blog posts as content for emails that prompted users to download and install a Microsoft Windows security system fix. Once downloaded, this malware would not only extract data but also installed additional software to crack and extract military documents. A total of 81 computers were found to be infected with around 1533 confidential/sensitive documents stolen. The main threat of spear phishing is how convincing the email/communication may seem from the point of view of the victim, often making use of a need for urgency and specific contextual knowledge \citep{nicho2018evaluating}.


\section{Effective Training}


\section{Related Systems}
Due to the popularity of phishing, many different attempts at anti-phishing training have been made before. From these, there are a few key systems that attempt to make use of both user interaction and context in phishing. The user interaction put foward by the following systems are key in effective training especially in avoid phishing attacks.

\subsection{Anti-Phishing Phil}
Anti-Phishing Phil \citep{sheng2007anti} is an online educational game that attempts to teach players how to successfully avoid phishing attacks. This is done mainly by teaching them how to identify red flags and tells given by phishing emails and URLs. The game itself gets the player to take the role of ‘Phil’ a fish who is attempting to eat as many worms as possible, as shown in Figure~\ref{fig:antiphishingphil}. Each worm is paired with a URL which is potentially malicious, and the player must decide whether to eat the worm/URL (therefore considering it to be legitimate) or reject the worm/URL (considering it to be a phishing URL). If the player is stuck, they can ask for help from Phil’s father who will give them hints as to the nature of URL. 

Although this may be helpful in teaching the key methods to detecting and avoiding phishing attacks, the complete lack of context takes away any benefits this training program might provide. As a result of this, the users will have difficulty applying the knowledge learned to any real-life situations. 

\subsection{What.Hack}
What.Hack \citep{wen2019hack} is student project to create an anti-phishing training game in which players have to discern phishing emails from legitimate emails. The game takes the form of an (extremely simplified) email client. Once the user has made a choice, they must click either the accept or reject buttons based on their guess. For additional help the user can ask an in-game advisor to give hints, as show in Figure~\ref{fig:whathack_1}. On top of the advisor, the players can access a ‘rulebook’, giving them a series of rules/guidelines to follow to determine if an email is malicious. This application applies a much greater level of context to the simulated phishing attacks, from the actual design of the user experience  right through to the content of the actual emails. 

Framing the game itself to take the form of an email client gives the player a greater frame of reference for what actual phishing attacks look like 'out in the wild'. However, the design of the client has been streamlined to the point of losing key elements of the contextual value of framing it as a email client. As shown in Figure~\ref{fig:whathack_2}, each of the emails contains 'Submit Phishing' and 'Allow Entry' buttons. For the regular phishing victim, these buttons do not exist within standard email clients and the 'gamification' of these elements only depreciates the important contextual value that a similar product might hold.

\subsection{PhishGuru}
PhishGuru \citep{kumaraguru2008lessons} was a training program run to train employees how to identify and avoid phishing attacks. The training scheme came in the form of an email which presented a short comic strip (Figure~\ref{fig:phishguru}) that briefly went over the key red flags of a phishing email. Once this short phishing comic has been delivered to the participants, they were (secretly) sent fake phishing emails to their workplace email accounts by the creators of PhishGuru. These emails would collect data and statistics of how the participant would deal with the email. 

Although this training scheme makes use of similar advice and tips for avoiding phishing, implementing it through a comic strip makes poor use of any training techniques. The complete lack of contextual information or user interaction with the training material makes this a poor attempt at best. This is even shown in the results in which only a very small difference between the control group and the group that received the training comic strip.

Despite this, the experiment to test the effectiveness of the training made excellent use of context. With each of the participant receiving a fake phishing email containing all of the phishing red flags, logging the actions made by the user when dealing with the email. The use of the participants actual email client under a blind test makes great use of ecological validity within phishing training. If this had been coupled with an improved initial training scheme, the validity of the results would be much greater.

\subsection{Summary of Existing Systems}
Despite all attempting anti-phishing training, all of these systems approach the activity in very different ways. All of the systems teach the user similar phishing tactics such as typo-squatting and sound-squatting. However, in terms of context, most of the systems had poor context integration, often being to simplified (in the case of What.Hack) or virtually non-existent (PhishGuru). This shows a lack of understanding as to the best training methods when it comes to phishing, that being the importance of context for the user. 

The methods of evaluation for each of these systems differed, with the game-based training programs using the in-game score as a basis for the knowledge gained for the user, and the PhishGuru making use of contextual blind phishing test emails. Although the evaluation made by the PhishGuru project is excellent in terms of its similarity to real-world phishing attacks, the time and effort required to implement a similar evaluation technique may be out of scope for this project. 

A table containing an overview of the different systems with a brief analysis can be seen in Figure~\ref{fig:sum_sys}.

\begin{table}[H]
\begin{adjustwidth}{-7em}{-5em}
\begin{tabular}{ | m{8em} | m{10em} | m{10em} | m{10em} | m{10em} | } 
 \hline
 \textbf{System} & \multicolumn{2}{l|}{\textbf{Phishing \& Context}} & \multicolumn{2}{l|}{\textbf{Training Effectiveness}}  \\
 \hline
   & \textbf{Pros} & \textbf{Cons} & \textbf{Pros} & \textbf{Cons} \\
 \hline
  Anti-Phishing Phil & Links are similar to those found in phishing emails & lacks ecological validity  & & Difficult to apply to real life phishing attacks 
  \\
 \hline
 What.Hack & Accurate representation of real phishing attacks & Simplified and 'gamified' user experience takes away from validity & variety of knowledge gained and easily transferable to 'real world' applications & 
  \\
 \hline
 PhishGuru & & Information given on phishing and attacks is extremely brief and uninformative & & Very little actual content to training and user interaction
  \\
 \hline
\end{tabular}
\end{adjustwidth}
\caption{Brief summary of existing systems}
\label{fig:sum_sys}
\end{table}


%==================================================================================================================================
\chapter{Analysis/Requirements}

\section{User Stories}
\textbf{Matt} is a bank manager and manages multiple teams within the department. Due to the rise in internet-based attacks, he wants to train his staff for the different types of attacks they may face, one of these being phishing attacks. Matt wants to give his staff regular anti-phishing training exercises that they can complete at work or at home. Due to workplace regulations and computer safety, he cannot install any software on the work computers so must have a training program that can run in a web browser. He wants to easily be able to see his employees progress, including what they are consistently missing/failing at and what their strengths are. 

\textbf{Thomas} is an employee in the software development department on company. They have plenty of technical experience and wish to keep up to date with the latest anti-phishing training techniques. He wants to be able to avoid the basic “tutorial” type training and go straight into the more technical, in-depth training. Thomas would be willing to install any type of training software as it would be easy for him to do. 

\textbf{Sophie} is an employee for the human resources department of a company. They have little to no technical experience outside of operating their email client and similar tasks. Due to recent phishing attacks, they have been instructed to learn about phishing and methods to prevent successful attacks. Due to their lack of experience, they need every aspect of phishing explained in detail for the training to work successfully. Due to the amount of emails she receives, she needs lots of varied examples of common signposts of phishing emails. Her lack of proficiency in technology means she requires something that does not need to be install or have any complex set-up, preferably something she can use in her browser.

\section{Functional Requirements}
\subsection{Must Have}
\begin{tabular}{ | m{10em} | m{25em}| m{5em} | } 
  \hline
  \textbf{Requirement} & \textbf{Description} & \textbf{Reference} \\ 
  \hline
  Training & This application must train the user in anti-phishing practices as per the product description &  \\ 
  \hline
  Fail States & The user must be able to pass or fail tasks based on their input as this would be effective training &  \\ 
  \hline
  Context in Design & The application must make effective use of context within the look/feel of the application itself as well as the tasks given to the user &  \\  
  \hline
  Context in Content & The application must make use of different levels of context within the actual content of the exercises the user needs to perform &  \\  
  \hline
  Broad representation & This application must display a broad range of different types of phishing to the user, teaching them methods to detect and avoid these different types &  \\  
  \hline
\end{tabular}

\subsection{Should Have}
\begin{tabular}{ | m{10em} | m{25em}| m{5em} | } 
  \hline
  \textbf{Requirement} & \textbf{Description} & \textbf{Reference} \\ 
  \hline
  Reflection & The application should be able to report the user’s progress through the training, including their weakest and strongest points &  \\ 
  \hline
  Immersion & The user should fine the application easy to immerse themselves in the game in terms of familiarity and comfortability  &  \\ 
  \hline
  Assistance & The application should assist the user as much as possible without giving them the answer (if the user is struggling with the task) &  \\ 
  \hline
\end{tabular}

\subsection{Could Have}
\begin{tabular}{ | m{10em} | m{25em}| m{5em} | } 
  \hline
  \textbf{Requirement} & \textbf{Description} & \textbf{Reference} \\ 
  \hline
  Difficulty Customisation & The application could give the user control over how difficult the training tasks are to allow more technically advanced users to jump ahead to more advanced training techniques &  \\ 
  \hline
  Multiple "Campaigns" & The application could make use of a number of “real-life” phishing attacks as training exercises  &  \\ 
  \hline
  Glossary & The application could give the user access to a glossary containing technical terms with descriptions and/or examples &  \\ 
  \hline
\end{tabular}

\subsection{Would Be Nice To Have}
\begin{tabular}{ | m{10em} | m{25em}| m{5em} | } 
  \hline
  \textbf{Requirement} & \textbf{Description} & \textbf{Reference} \\ 
  \hline
  User History & The application would benefit from a form of training history for each user to view progress over time &  \\ 
  \hline
\end{tabular}

\section{Non-Functional Requirements}
\begin{tabular}{ | m{10em} | m{25em}| m{5em} | } 
  \hline
  \textbf{Requirement} & \textbf{Description} & \textbf{Reference} \\ 
  \hline
  Ease of Installation & The application should be easy to install/use on a broad range of devices and environments (e.g. workplace, home PC, etc) &  \\ 
  \hline
  Multiple Devices & The application should work across a broad range on devices including tablets and mobile phones &  \\ 
  \hline
  Accessibility & The application should train people at any knowledge level and level of technical skill &  \\ 
  \hline
  Speed & The application should be fast to boot and operate &  \\ 
  \hline
\end{tabular} 

%==================================================================================================================================
\chapter{Design}
\section{Initial System Architecture}
In order to outline the system to implement, a system diagram needs to be created. As you can see below, the system architecture diagram is a three-tier diagram for each of the key elements of the application: client tier, logic tier, and database tier \citep{aarsten1996patterns}. 

\begin{figure}[H]
    \centering
    \includegraphics[width=0.9\linewidth]{images/initial_arch.png}   
    \caption{Initial system architecture outlining basic interactions between core systems}
    \label{fig:initial_arch} 
\end{figure}

The client tier of the diagram gives an overview of the user interface of the application, including the different pages the user can view. The logic tier details the application backend interface or general logic. The database tier details the location of all relevant application data storage.
\newline

Between each of the tiers, there are indicators of incoming and outgoing communication between the tiers. With the client tier communication exclusively with the logic tier, and the logic tier interaction with both the client tier and the database tier. At no point does the database tier directly interact with the client tier.
\newline

The three-tier model was appropriate for this project due to the scalability and reliability of its fundamental design. Each tier is independent enough to be scaled as needed without too much impact on the other tiers. On top of this, if a single tier starts performing badly, the other tiers will not be impacted by this. All of this leads to a more efficient and faster development time.

\section{Training as a Game}
[talk about use of game mechanics as effective training]

\section{Context in Design}
The user interface for this training game must take into account key usability heuristics \citep{nielsen2005ten}. These heuristics cover a varity of principles concerning the design of user interfaces, but there are several in which play a larger part in this project. Namely, the second and sixth heuristics. The second principle is the idea that there should be a match between the system design and world of the user, that being one in which they understand all the facets of. For this project that is key in the application of context and ecological validity in terms of the transfer of learnt knowledge from the training being easily applicable to a real phishing attack. The sixth principle promotes the concept of users being able to intuitively navigate and access the program with minimal (or no) instructions. This is key in giving the user autonomy in moving up a level or replaying sections of the game.

The design of the application is key to immersing the user into an accurate simulation of a phishing attack, this being an email client. There are numerous email clients on the market, the most popular being Google Mail, Microsoft Outlook, and Apple Mail \citep{emailclientlitmus}. For the Apple and Outlook clients, a similar approach in design is taken, with the list of emails on the right hand side and the full selected email display on the left. With the Google Mail client, the email list is more vertically condensed per email but takes up the entire page. Once an email is clicked, then that selected email will take up the whole page. Above these sections is a navigation bar which gives contextual options to the user (often related to the selected email). 

For this application, I will attempt to combine components of the Apple Mail (Figure~\ref{fig:applemail}) and Outlook (Figure~\ref{fig:outlook}) clients as these give great overall visibility for emails. From the possible options, Microsoft Outlook was chosen as the base design of the application. This was mainly chosen due to a higher level of experience with Outlook over the other possible clients. This design will effect the colour scheme and layout of outlook.

This context should also apply to the game mechanics themselves. In this sense, the player/user would simply operate the email client as they would a normal client. Therefore 'gameified' terms will not be used in the actual content of the application, for example incrememnting the levels will not be displayed to the user as a "next level" button but rather something that would make sense in the context of an email client such as "get mail". This will not only allow users to easily make use of the game but also have a more effective integration with real phishing attacks.

\section{Level Progression}



\section{Wireframes and Features}
At the core design, there are several main views: the login view, the main email view without email displayed, and the main email view with an email displayed. For each of the proposed views, wireframes were created in order to hone a standardised design pattern. For the sake of user immersion and econological validity, the design of all of these screens will be in the format of an email. This includes content that could be considered immersion "breaking", such as the results screen. The user will never encounter an email like this in the real world but the consistency fo this design is paramount for the training to be effective. 


\subsection{Main View}

This is the main view that the users will spend the majority of their time navigating. In this screen, users will have a series of emails arrive, to which they need to review/analyse each email to determine whether the email is a phishing email or not. For each of the emails, the user will have a choice as to whether they should accept the email (if the user deems it as non-malicious) or if they should reject it (if the user considers it to be a phishing email). Once all of the emails in the selection have been either accepted or rejected, the user is given the option to review their decisions and progress to the next set of emails. As to the content of the emails, each email will contain text similar to those found in regular workplace emails, occasionally sporting attachments and links. However, for each of these groups of emails, several will be phishing emails, making use of the numerous phishing tactics.

\begin{figure}[H]
    \centering
    \frame{\includegraphics[width=1\linewidth]{images/main_screen_wf.png}}
    \caption{Wireframe of the main play screen}
    \label{fig:main_wf} 
\end{figure}

\subsection{Results View}
As previously stated, an important part of training is giving feedback to the user giving them the opportunity to review their progress. On this screen, the user will be presented with the number of correctly identified emails (rejected phishing emails and allowed non-phishing emails) and incorrect guesses. There is also a breakdown of the incorrect guesses, that being the number of false positives (accepting phishing emails) and false negatives (rejecting non-phishing emails). From this screen, the user has two options: continue to the next level or repeat the current level in order to correct their mistakes. 

\begin{figure}[H]
    \centering
    \frame{\includegraphics[width=1\linewidth]{images/results_wf.png}}
    \caption{Wireframe of the end of level results screen}
    \label{fig:results_wf} 
\end{figure}

\subsection{Menus \& Additional Views}
The main menu (as shown in Figure~\ref{fig:menu_wf}) will provide the user with numerous options for both new and returning users. New users will be prompted to take the tutorial which will guide the user on how to "play" the training game and key aspects of the interface. For each of the levels there will be a button to instantly start the game at the specified level.

\begin{figure}[H]
    \centering
    \frame{\includegraphics[width=1\linewidth]{images/menu_wf.png}}
    \caption{Wireframe of main menu screen}
    \label{fig:menu_wf} 
\end{figure}

%==================================================================================================================================
\chapter{Implementation}

\section{Application Base}
As a result of the descision to emulate Microsoft's Outlook email client, the obvious choice for development/design would be a web-based application. This not only fulfills the requirements set out by the user stories in terms of ease of installation and accessible from multiple device types, but would also be accurate to the core philosophy of emulating an environment in which phishing attacks occur.

\section{Technologies}

\section{Site Map}

\section{Development}

\section{Changes Made}

\section{Challenges Faced}

\section{Final Product}





What did you do to implement this idea, and what technical achievements did you make?
\section{Guidance}
You can't talk about everything. Cover the high level first, then cover important, relevant or impressive details.

\section{General points}

These points apply to the whole dissertation, not just this chapter.



\subsection{Figures}
\emph{Always} refer to figures included, like Figure \ref{fig:relu}, in the body of the text. Include full, explanatory captions and make sure the figures look good on the page.
You may include multiple figures in one float, as in Figure \ref{fig:synthetic}, using \texttt{subcaption}, which is enabled in the template.



% Figures are important. Use them well.
\begin{figure}
    \centering
    \includegraphics[width=0.5\linewidth]{images/relu.pdf}    

    \caption{In figure captions, explain what the reader is looking at: ``A schematic of the rectifying linear unit, where $a$ is the output amplitude,
    $d$ is a configurable dead-zone, and $Z_j$ is the input signal'', as well as why the reader is looking at this: 
    ``It is notable that there is no activation \emph{at all} below 0, which explains our initial results.'' 
    \textbf{Use vector image formats (.pdf) where possible}. Size figures appropriately, and do not make them over-large or too small to read.
    }

    % use the notation fig:name to cross reference a figure
    \label{fig:relu} 
\end{figure}


\begin{figure}
    \centering
    \begin{subfigure}[b]{0.45\textwidth}
        \includegraphics[width=\textwidth]{images/synthetic.png}
        \caption{Synthetic image, black on white.}
        \label{fig:syn1}
    \end{subfigure}
    ~ %add desired spacing between images, e. g. ~, \quad, \qquad, \hfill etc. 
      %(or a blank line to force the subfigure onto a new line)
    \begin{subfigure}[b]{0.45\textwidth}
        \includegraphics[width=\textwidth]{images/synthetic_2.png}
        \caption{Synthetic image, white on black.}
        \label{fig:syn2}
    \end{subfigure}
    ~ %add desired spacing between images, e. g. ~, \quad, \qquad, \hfill etc. 
    %(or a blank line to force the subfigure onto a new line)    
    \caption{Synthetic test images for edge detection algorithms. \subref{fig:syn1} shows various gray levels that require an adaptive algorithm. \subref{fig:syn2}
    shows more challenging edge detection tests that have crossing lines. Fusing these into full segments typically requires algorithms like the Hough transform.
    This is an example of using subfigures, with \texttt{subref}s in the caption.
    }\label{fig:synthetic}
\end{figure}

\clearpage

\subsection{Equations}

Equations should be typeset correctly and precisely. Make sure you get parenthesis sizing correct, and punctuate equations correctly 
(the comma is important and goes \textit{inside} the equation block). Explain any symbols used clearly if not defined earlier. 

For example, we might define:
\begin{equation}
    \hat{f}(\xi) = \frac{1}{2}\left[ \int_{-\infty}^{\infty} f(x) e^{2\pi i x \xi} \right],
\end{equation}    
where $\hat{f}(\xi)$ is the Fourier transform of the time domain signal $f(x)$.

\subsection{Algorithms}
Algorithms can be set using \texttt{algorithm2e}, as in Algorithm \ref{alg:metropolis}.

% NOTE: line ends are denoted by \; in algorithm2e
\begin{algorithm}
    \DontPrintSemicolon
    \KwData{$f_X(x)$, a probability density function returing the density at $x$.\; $\sigma$ a standard deviation specifying the spread of the proposal distribution.\;
    $x_0$, an initial starting condition.}
    \KwResult{$s=[x_1, x_2, \dots, x_n]$, $n$ samples approximately drawn from a distribution with PDF $f_X(x)$.}
    \Begin{
        $s \longleftarrow []$\;
        $p \longleftarrow f_X(x)$\;
        $i \longleftarrow 0$\;
        \While{$i < n$}
        {
            $x^\prime \longleftarrow \mathcal{N}(x, \sigma^2)$\;
            $p^\prime \longleftarrow f_X(x^\prime)$\;
            $a \longleftarrow \frac{p^\prime}{p}$\;
            $r \longleftarrow U(0,1)$\;
            \If{$r<a$}
            {
                $x \longleftarrow x^\prime$\;
                $p \longleftarrow f_X(x)$\;
                $i \longleftarrow i+1$\;
                append $x$ to $s$\;
            }
        }
    }
    
\caption{The Metropolis-Hastings MCMC algorithm for drawing samples from arbitrary probability distributions, 
specialised for normal proposal distributions $q(x^\prime|x) = \mathcal{N}(x, \sigma^2)$. The symmetry of the normal distribution means the acceptance rule takes the simplified form.}\label{alg:metropolis}
\end{algorithm}

\subsection{Tables}

If you need to include tables, like Table \ref{tab:operators}, use a tool like https://www.tablesgenerator.com/ to generate the table as it is
extremely tedious otherwise. 

\begin{table}[]
    \caption{The standard table of operators in Python, along with their functional equivalents from the \texttt{operator} package. Note that table
    captions go above the table, not below. Do not add additional rules/lines to tables. }\label{tab:operators}
    %\tt 
    \rowcolors{2}{}{gray!3}
    \begin{tabular}{@{}lll@{}}
    %\toprule
    \textbf{Operation}    & \textbf{Syntax}                & \textbf{Function}                            \\ %\midrule % optional rule for header
    Addition              & \texttt{a + b}                          & \texttt{add(a, b)}                                    \\
    Concatenation         & \texttt{seq1 + seq2}                    & \texttt{concat(seq1, seq2)}                           \\
    Containment Test      & \texttt{obj in seq}                     & \texttt{contains(seq, obj)}                           \\
    Division              & \texttt{a / b}                          & \texttt{div(a, b) }  \\
    Division              & \texttt{a / b}                          & \texttt{truediv(a, b) } \\
    Division              & \texttt{a // b}                         & \texttt{floordiv(a, b)}                               \\
    Bitwise And           & \texttt{a \& b}                         & \texttt{and\_(a, b)}                                  \\
    Bitwise Exclusive Or  & \texttt{a \textasciicircum b}           & \texttt{xor(a, b)}                                    \\
    Bitwise Inversion     & \texttt{$\sim$a}                        & \texttt{invert(a)}                                    \\
    Bitwise Or            & \texttt{a | b}                          & \texttt{or\_(a, b)}                                   \\
    Exponentiation        & \texttt{a ** b}                         & \texttt{pow(a, b)}                                    \\
    Identity              & \texttt{a is b}                         & \texttt{is\_(a, b)}                                   \\
    Identity              & \texttt{a is not b}                     & \texttt{is\_not(a, b)}                                \\
    Indexed Assignment    & \texttt{obj{[}k{]} = v}                 & \texttt{setitem(obj, k, v)}                           \\
    Indexed Deletion      & \texttt{del obj{[}k{]}}                 & \texttt{delitem(obj, k)}                              \\
    Indexing              & \texttt{obj{[}k{]}}                     & \texttt{getitem(obj, k)}                              \\
    Left Shift            & \texttt{a \textless{}\textless b}       & \texttt{lshift(a, b)}                                 \\
    Modulo                & \texttt{a \% b}                         & \texttt{mod(a, b)}                                    \\
    Multiplication        & \texttt{a * b}                          & \texttt{mul(a, b)}                                    \\
    Negation (Arithmetic) & \texttt{- a}                            & \texttt{neg(a)}                                       \\
    Negation (Logical)    & \texttt{not a}                          & \texttt{not\_(a)}                                     \\
    Positive              & \texttt{+ a}                            & \texttt{pos(a)}                                       \\
    Right Shift           & \texttt{a \textgreater{}\textgreater b} & \texttt{rshift(a, b)}                                 \\
    Sequence Repetition   & \texttt{seq * i}                        & \texttt{repeat(seq, i)}                               \\
    Slice Assignment      & \texttt{seq{[}i:j{]} = values}          & \texttt{setitem(seq, slice(i, j), values)}            \\
    Slice Deletion        & \texttt{del seq{[}i:j{]}}               & \texttt{delitem(seq, slice(i, j))}                    \\
    Slicing               & \texttt{seq{[}i:j{]}}                   & \texttt{getitem(seq, slice(i, j))}                    \\
    String Formatting     & \texttt{s \% obj}                       & \texttt{mod(s, obj)}                                  \\
    Subtraction           & \texttt{a - b}                          & \texttt{sub(a, b)}                                    \\
    Truth Test            & \texttt{obj}                            & \texttt{truth(obj)}                                   \\
    Ordering              & \texttt{a \textless b}                  & \texttt{lt(a, b)}                                     \\
    Ordering              & \texttt{a \textless{}= b}               & \texttt{le(a, b)}                                     \\
    % \bottomrule
    \end{tabular}
    \end{table}
\subsection{Code}

Avoid putting large blocks of code in the report (more than a page in one block, for example). Use syntax highlighting if possible, as in Listing \ref{lst:callahan}.

\begin{lstlisting}[language=python, float, caption={The algorithm for packing the $3\times 3$ outer-totalistic binary CA successor rule into a 
    $16\times 16\times 16\times 16$ 4 bit lookup table, running an equivalent, notionally 16-state $2\times 2$ CA.}, label=lst:callahan]
    def create_callahan_table(rule="b3s23"):
        """Generate the lookup table for the cells."""        
        s_table = np.zeros((16, 16, 16, 16), dtype=np.uint8)
        birth, survive = parse_rule(rule)

        # generate all 16 bit strings
        for iv in range(65536):
            bv = [(iv >> z) & 1 for z in range(16)]
            a, b, c, d, e, f, g, h, i, j, k, l, m, n, o, p = bv

            # compute next state of the inner 2x2
            nw = apply_rule(f, a, b, c, e, g, i, j, k)
            ne = apply_rule(g, b, c, d, f, h, j, k, l)
            sw = apply_rule(j, e, f, g, i, k, m, n, o)
            se = apply_rule(k, f, g, h, j, l, n, o, p)

            # compute the index of this 4x4
            nw_code = a | (b << 1) | (e << 2) | (f << 3)
            ne_code = c | (d << 1) | (g << 2) | (h << 3)
            sw_code = i | (j << 1) | (m << 2) | (n << 3)
            se_code = k | (l << 1) | (o << 2) | (p << 3)

            # compute the state for the 2x2
            next_code = nw | (ne << 1) | (sw << 2) | (se << 3)

            # get the 4x4 index, and write into the table
            s_table[nw_code, ne_code, sw_code, se_code] = next_code

        return s_table

\end{lstlisting}

%==================================================================================================================================
\chapter{Evaluation} 
How good is your solution? How well did you solve the general problem, and what evidence do you have to support that?

\section{Guidance}
\begin{itemize}
    \item
        Ask specific questions that address the general problem.
    \item
        Answer them with precise evidence (graphs, numbers, statistical
        analysis, qualitative analysis).
    \item
        Be fair and be scientific.
    \item
        The key thing is to show that you know how to evaluate your work, not
        that your work is the most amazing product ever.
\end{itemize}

\section{Evidence}
Make sure you present your evidence well. Use appropriate visualisations, reporting techniques and statistical analysis, as appropriate.

If you visualise, follow the basic rules, as illustrated in Figure \ref{fig:boxplot}:
\begin{itemize}
\item Label everything correctly (axis, title, units).
\item Caption thoroughly.
\item Reference in text.
\item \textbf{Include appropriate display of uncertainty (e.g. error bars, Box plot)}
\item Minimize clutter.
\end{itemize}

See the file \texttt{guide\_to\_visualising.pdf} for further information and guidance.

\begin{figure}
    \centering
    \includegraphics[width=1.0\linewidth]{images/boxplot_finger_distance.pdf}    

    \caption{Average number of fingers detected by the touch sensor at different heights above the surface, averaged over all gestures. Dashed lines indicate
    the true number of fingers present. The Box plots include bootstrapped uncertainty notches for the median. It is clear that the device is biased toward 
    undercounting fingers, particularly at higher $z$ distances.
    }

    % use the notation fig:name to cross reference a figure
    \label{fig:boxplot} 
\end{figure}


%==================================================================================================================================
\chapter{Conclusion}    
Summarise the whole project for a lazy reader who didn't read the rest (e.g. a prize-awarding committee).
\section{Guidance}
\begin{itemize}
    \item
        Summarise briefly and fairly.
    \item
        You should be addressing the general problem you introduced in the
        Introduction.        
    \item
        Include summary of concrete results (``the new compiler ran 2x
        faster'')
    \item
        Indicate what future work could be done, but remember: \textbf{you
        won't get credit for things you haven't done}.
\end{itemize}

%==================================================================================================================================
%
% 
%==================================================================================================================================
%  APPENDICES  

\begin{appendices}

\chapter{Appendices}

Typical inclusions in the appendices are:

\begin{itemize}
\item
  Copies of ethics approvals (required if obtained)
\item
  Copies of questionnaires etc. used to gather data from subjects.
\item
  Extensive tables or figures that are too bulky to fit in the main body of
  the report, particularly ones that are repetitive and summarised in the body.

\item Outline of the source code (e.g. directory structure), or other architecture documentation like class diagrams.

\item User manuals, and any guides to starting/running the software.

\end{itemize}

\textbf{Don't include your source code in the appendices}. It will be
submitted separately.

\begin{figure}
    \centering
    \includegraphics[width=1.1\linewidth]{images/whathack_1.png}    
    \caption{What.Hack main game screen with advisor giving player tips on the contents of the email, pointing out the use of typo-squatting and URL manipulation}
    \label{fig:whathack_1} 
\end{figure}
\begin{figure}
    \centering
    \includegraphics[width=1.1\linewidth]{images/whathack_2.png}    
    \caption{What.Hack main game screen with showing the buttons and available actions given to the player}
    \label{fig:whathack_2} 
\end{figure}
\begin{figure}
    \centering
    \includegraphics[width=0.9\linewidth]{images/antiphishingphil.png}   
    \caption{Anti-Phishing Phil game screen with the player fish interacting with a worm holding a URL, with the fish advisor giving advice to the player}
    \label{fig:antiphishingphil} 
\end{figure}
\begin{figure}
    \centering
    \includegraphics[width=1.1\linewidth]{images/phishguru.png}    
    \caption{The PhishGuru comic strip giving general tips to avoid phishing attacks }
    \label{fig:phishguru} 
\end{figure}

\begin{figure}
    \centering
    \includegraphics[width=1.1\linewidth]{images/outlook.png}
    \caption{Outlook email client}
    \label{fig:outlook}
\end{figure}

\begin{figure}
    \includegraphics[width=1.1\linewidth]{images/applemail.png}
    \caption{Apple Mail desktop email client}
    \label{fig:applemail}
\end{figure}


\end{appendices}

%==================================================================================================================================
%   BIBLIOGRAPHY   

% The bibliography style is abbrvnat
% The bibliography always appears last, after the appendices.

\bibliographystyle{abbrvnat}

\bibliography{l4proj}

\end{document}
